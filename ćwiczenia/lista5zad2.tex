%&pdflatex
\documentclass[10pt,a4paper]{article}
\usepackage{geometry}
\geometry{margin=1in}
\usepackage[utf8]{inputenc}
\usepackage{polski}
\usepackage[polish]{babel}
\usepackage{amsmath}
\usepackage{listings}
\lstset{
	basicstyle=\small\ttfamily,
	columns=flexible,
	breaklines=true,
	frame=single
}
\usepackage{algpseudocode}
\usepackage{adjustbox}
\usepackage{graphicx}
\usepackage{tabularx}
\usepackage{color}
\usepackage{enumitem}

\begin{document}

\title{%
	Języki formalne i techniki translacji\\
	\large Lista 5, zadanie 2}
\author{Jędrzej Ginter, 204420}
\date{16 stycznia 2017}
\maketitle

\section{Treść problemu}
Pokazać, że jeśli $L$ jest językiem bezkontekstowym, to istnieje automat ze stosem $M$ akceptujący $L$ przy stanie końcowym, taki, że $M$ ma co najwyżej dwa stany i nie wykonuje $\epsilon$-ruchów.

\section{Rozwiązanie}
Język $L$ jest bezkontekstowy, więc $\exists G$, taka, że $L(G) = L$.\\
Załóżmy, że gramatyka $G$ jest w postaci Greibach:
\[ p \in P : A \rightarrow a\Gamma, \]
\[ A \in N, a \in T, \Gamma = N* \]
Z pewnego lematu wiemy, że jeśli $L$ jest bezkontektstowy i $\epsilon \notin L$, to istnieje PDA $M$ taki, że $L(M) = L = L(G)$ zdefiniowany w następujący sposób:
\[ M = (\{q\}, \Gamma, N, \delta, q, S, \emptyset) \]
\[ \forall p : A \rightarrow a\gamma \] \[\delta(q, a, A) = \{(q, \gamma)\} \]
\[ \forall p : A \rightarrow a \] \[\delta(q, a, A) = \{(q, \epsilon)\} \]
Jeżeli $\epsilon \in L$, to dodaje się nowy symbol $S\prime$ do gramatyki i dodatkową produkcję $S \rightarrow \epsilon \vert S$. W automacie należy dodać przejście $\delta(q, \epsilon, S\prime) = \{(q, \epsilon),(q, S)\}$.
Mamy jednak manipulacje stosem przy pustym wejściu ($\epsilon$-ruchy). Należy pozbyć się $\epsilon$-ruchów wykorzystując fakt, że automat może być dwustanowy. W tym celu wprowadzamy nowe symbole stosu:
\[ \forall A \in N \]
\[ A, \overline{A} \in \Gamma \]
\[ \Gamma = \{A, \overline{A}, A \in N\} \]
dla produkcji $ A \rightarrow a\gamma$, $\gamma = \gamma_{1},\gamma_{2}, ..., \gamma_{n},$ $\gamma_{i} \in N $ wprowadzamy przejścia:
\[ \delta(q, a, A) = \{(q, \gamma)\}\]
\[ \delta(q, a, \overline{A}) = \{(q,  \gamma_{1}...\gamma_{n-1}\overline{\gamma_{n}})\} \]
dla produkcji $A \rightarrow a$:
\[ \delta(q, a, A) = \{(q, \epsilon)\} \]
\[ \delta(q, a, \overline{A}) = \{(qF , \epsilon)\} \]
oraz ustawiamy $\overline{S}$ jako symbol początkowy stosu.

\end{document}
