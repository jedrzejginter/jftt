%&pdflatex
\documentclass[10pt,a4paper]{article}
\usepackage{geometry}
\geometry{margin=1in}
\usepackage[utf8]{inputenc}
\usepackage{polski}
\usepackage[polish]{babel}
\usepackage{amsmath}
\usepackage{listings}
\lstset{
	basicstyle=\small\ttfamily,
	columns=flexible,
	breaklines=true,
	frame=single
}
\usepackage{algpseudocode}
\usepackage{adjustbox}
\usepackage{graphicx}
\usepackage{tabularx}
\usepackage{color}
\usepackage{enumitem}

\begin{document}

\title{%
	Języki formalne i techniki translacji\\
	\large Lista 6, zadanie 4}
\author{Jędrzej Ginter, 204420}
\date{16 stycznia 2017}
\maketitle

\section{Treść problemu}
Podać definicję sterowaną składnią dla rozważanej gramatyki:\\

{\renewcommand{\arraystretch}{1.25}%
	\begin{adjustbox}{center}
		\begin{tabular}{l}
			E $\rightarrow$ E or T $\vert$ T\\
			T $\rightarrow$ T and F $\vert$ F\\
			F $\rightarrow$ not F $\vert$ (E) $\vert$ true $\vert$ false\\
		\end{tabular}
	\end{adjustbox}
}\quad

\section{Rozwiązanie}
Należy rozbić wyprowadzenia gramatyki na pojedyncze i zapisać definicję sterowaną składnią dla otrzymanych wyprowadzeń.

\subsection{Rozbicie wyprowadzeń gramatyki}
{\renewcommand{\arraystretch}{1.25}%
	\begin{adjustbox}{center}
		\begin{tabular}{l}
			E $\rightarrow$ E or T\\
			E $\rightarrow$ T\\
			T $\rightarrow$ T and F\\
			T $\rightarrow$ F\\
			F $\rightarrow$ not F\\
			F $\rightarrow$ (E)\\
			F $\rightarrow$ true\\
			F $\rightarrow$ false\\
	\end{tabular}
	\end{adjustbox}
}\quad

\subsection{Definicja sterowana składnią}
{\renewcommand{\arraystretch}{1.25}%
	\begin{adjustbox}{center}
		\begin{tabular}{ll}
			E $\rightarrow$ E$_{*}$ or T & E.wart = (E$_{*}$.wart \textit{OR} T.wart) ? true : false \\
			E $\rightarrow$ T & E.wart = T.wart \\
			T $\rightarrow$ T$_{*}$ and F & T.wart = (T$_{*}$.wart \textit{AND} F.wart) ? true : false \\
			T $\rightarrow$ F & T.wart = F.wart \\
			F $\rightarrow$ not F$_{*}$ & F.wart = (F$_{*}$.wart) ? false : true \\
			F $\rightarrow$ (E) & F.wart = E.wart \\
			F $\rightarrow$ true & F.wart = true \\
			F $\rightarrow$ false & F.wart = false\\
		\end{tabular}
	\end{adjustbox}
} \quad

\textit{Niejednoznaczność została wyeliminowana poprzez nadanie nieterminalom po prawej stronie indeksów dolnych.}\\

\textit{Oznaczenie (A)? B : C jest równoważne wyrażeniu: if A then B else C.}

\end{document}
